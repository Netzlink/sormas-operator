\documentclass[a4paper,11pt]{article}
\usepackage[T1]{fontenc}
\usepackage[ngerman]{babel}
\title{
  Entwicklung eines Operator zur Installation und 
  \textit{2
  \textsuperscript{nd} day operations
  } von Sormasinstanzen auf Kubernetes.
}
\date{07.07.2020}
\author{Nico Kahlert}
\begin{document}
  \maketitle
  \newpage
  \tableofcontents
  \newpage
  \pagenumbering{arabic}
  \vspace{2.5cm}
  \section{Einleitung}
    \subsection{Vortellung des Kunden}
    Die Netzlink Informationstechnik GmbH ist ein IT-Systemhaus mit ca. 90
    Mitarbeitern. Zur Zielgruppe des Unternehmens gehören hauptsächlich 
    Kundes aus dem Mittelstand, für welche IT-Dienstleistungen On-Premise oder in der Cloud
    erbracht werden. Die drei Firmenstandorte befinden sich in Braunschweig, Kassel und Hannover.
    Außerdem führt Netzlink drei georedundante Rechenzentren in Hannover, Salzgitter und Braunschweig.
    Jene dienen sowohl der eigenen Infrastruktur, als auch Kundenprojekten. 
    \subsection{Auswahl des Projektes}
    Die bis zum Verfassungszeitpunkt andauernde Krise um die Pandemie des SARS-Cov-2 Erregers 
    in den Jahren 2019/2020, hat einen Bedarf an Software zur zentralen Dokumentation und Analyse 
    einer Epidemie ausgelöst. Um das Cloudportfolio der neuen Herausforderung anzupassen, wurde die Sormas-
    software als SAAS Lösung aufgenommen. \newline >>///<<Hier erweitern>>///<<
      \subsubsection{Sormas}
      Sormas ist eine eine quelloffene Softwarelösung zum Management und der Analyse von Epidemien.
      Entwickler ist die Firma Symeda und das Helmholtzzentrum für Infektionsforschung Braunschweig.
      Der ursprüngliche Einsatzzweck von Sormas war die Eindämmung und das Management der 
      Ebolaepidemie in Westafrika im Jahr 2014, wo sie bis heute eingesetzt wird.
      \newline Die Lösung besteht im großenganzen aus vier Komponenten: Einem relationalen 
      Postgresdatenbankmanagementsystems in welchem der gesamte Datenstand gehalten wird. Einem 
      Optionalen Mailserver zum Verschicken von E-Mailbenachrichtigungen. Einem Hazelcast-LRU-Cache
      ,welcher in den Payara eingebettet oder extern betrieben werden kann und zur temporären Speicherung
      der Nutzersitzungen dient. Und zuletzt einem Javaapplikationsserver von Payara auf welchem Javaapplikationen 
      ausgeführt werden. Bei Sormas wurden zwei Javaservlet entwickelt, welche die Applikationslogik beinhalten.
      Das \textit{web-ui} Servlet stellt eine Javascript basierte, grafische Nutzerschnittstelle  über das Web 
      bereit an dem der Nutzer Plattformunabhängig arbeiten kann. Außerdem gibt es ein \textit{rest} Servlet, welches
      eine REST-API für die Apps mobiler Endgeräte bereitstellt. Zur Installation auf einem Host werden 
      Containerimages und dazugehörige Konfigurationsdateinen von Netzlink gebaut und auf Github angeboten.
      Sormas und seine Komponenten sind komplett quelloffene Software, so ist eine Transparenz gegenüber dem
      Kunden gewährleistet. Die Lizenzen der einzenen Projekte sind jedoch teilweise absent und somit proprietär.
      \subsubsection{RedHat OpenShift Container PLattform}
      Die OpenShift Container Plattform ist eine \textit{enterprise} Kubernetesdistribution der Firma RedHat. Neben den
      Funktionalitäten von Kubernetes beinhaltet die Plattform einen Supportvertrag und verbesserungen im Bereich 
      Lifecyclemanagement und Administrationsaufwand auf Kosten der Leichgewichtigkeit.   
      \subsubsection{Golang}
      \subsubsection{Operator Framework}
    \subsection{Wirtschaftliche Betrachtung}
  \section{Projektplanung}
    \subsection{Dokumentation und Management des Projekts}
    \subsection{Durchführung der IST-Analyse}
    \subsection{Ermittlung des SOLL-Zustands}
      \subsubsection{Evaluierung der Betriebsplattform}
      \subsubsection{Evaluierung der Installationswerkzeuge}
  \section{Projektdurchführung}
    \subsection{Ermittlung der Konfigurationspunkte der Sormascontainer}
    \subsection{Ermittlung der Zielkonfiguration einer Sormasinstallation}
    \subsection{Programmierung des Operators}
      \subsubsection{Einrichten eines Repositories und Intitialisierung}
      \subsubsection{Erstellen einer neuen API Ressource}
      \subsubsection{Implementierung der Konfiguration in API Ressource}
      \subsubsection{Implementierung des Controllers der API Ressource}
      \subsubsection{Durchführung von lokalen Tests}
    \subsection{Installation des Operators}
      \subsubsection{Deployment via CLI}
      \subsubsection{Instaziierung eines Sormas}
  \section{Fazit}
\end{document}