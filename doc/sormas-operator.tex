\documentclass[a4paper,11pt]{article}
\usepackage[T1]{fontenc}
\usepackage[ngerman]{babel}
\title{
  Entwicklung eines Operator zur Installation und 
  \textit{2
  \textsuperscript{nd} day operations
  } von Sormasinstanzen auf Kubernetes.
}
\date{07.07.2020}
\author{Nico Kahlert}
\begin{document}
  \maketitle
  \newpage
  \tableofcontents
  \newpage
  \pagenumbering{arabic}
  \vspace{2.5cm}
  \section{Einleitung}
    \subsection{Vortellung des Kunden}
    Die Netzlink Informationstechnik GmbH ist ein IT-Systemhaus mit ca. 90
    Mitarbeitern. Zur Zielgruppe des Unternehmens gehören hauptsächlich 
    Kundes aus dem Mittelstand, für welche IT-Dienstleistungen On-Premise oder in der Cloud
    erbracht werden. Die drei Firmenstandorte befinden sich in Braunschweig, Kassel und Hannover.
    Außerdem führt Netzlink drei georedundante Rechenzentren in Hannover, Salzgitter und Braunschweig.
    Jene dienen sowohl der eigenen Infrastruktur, als auch Kundenprojekten. 
    \subsection{Auswahl des Projektes}
    Die bis zum Verfassungszeitpunkt andauernde Krise um die Pandemie des SARS-Cov-2 Erregers 
    in den Jahren 2019/2020, hat einen Bedarf an Software zur zentralen Dokumentation und Analyse 
    einer Epidemie ausgelöst.  
      \subsubsection{Sormas}
      \subsubsection{RedHat OpenShift Container PLattform}
      \subsubsection{Golang}
      \subsubsection{Operator Framework}
    \subsection{Wirtschaftliche Betrachtung}
  \section{Projektplanung}
    \subsection{Dokumentation und Management des Projekts}
    \subsection{Durchführung der IST-Analyse}
    \subsection{Ermittlung des SOLL-Zustands}
      \subsubsection{Evaluierung der Betriebsplattform}
      \subsubsection{Evaluierung der Installationswerkzeuge}
  \section{Projektdurchführung}
    \subsection{Ermittlung der Konfigurationspunkte der Sormascontainer}
    \subsection{Ermittlung der Zielkonfiguration einer Sormasinstallation}
    \subsection{Programmierung des Operators}
      \subsubsection{Einrichten eines Repositories und Intitialisierung}
      \subsubsection{Erstellen einer neuen API Ressource}
      \subsubsection{Implementierung der Konfiguration in API Ressource}
      \subsubsection{Implementierung des Controllers der API Ressource}
      \subsubsection{Durchführung von lokalen Tests}
    \subsection{Installation des Operators}
      \subsubsection{Deployment via CLI}
      \subsubsection{Instaziierung eines Sormas}
  \section{Fazit}
\end{document}